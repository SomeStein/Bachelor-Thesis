%!TEX root = ../thesis.tex


\section{Introduction}

Evacuation of pedestrians from hazardous locations is an issue of great importance. 
Preventing unwanted outcomes starts by the architectural design of the area.
Obsticals and hidden exit doors may lead to injuries and even death in the attempt to escape.
But also the behaviour of the crowd itself can have a bad impact on the time it takes for everyone to evacuate.
But how can we figure out a way how for example a stadium needs to be build in the first place so that such casualties can be prevented.
This question leads to the research of Pedestrian Dynamics. 
Once a stable model can be derived it can be used to simulate such crowding or evacuation scenarios and as a goal 
become part of the designing and engeneering process of buildings, streets and parks itself.

One big issue with that is the sheer size of data a programm would have to consider. 
At this point in time it is just not feasible to simulate every stone and leaf physically accurate
so there has to be a step of nondimensionalization for it to work. 
This is part of the modelling process. 
Before choosing a model one has to be clear about what scaling and complexitiy is most suitable for the given usecase.
Some of them view the pedestrians homogenious (individually) others heterogenious (groups).
The scaling describes different levels of abstraction of the behaviour.
A microscopic scale gives information of the exact location of every particle/individual/agent of the system 
whilst a macroscopic scale tells more about the overall flow or density. There are also models that use scales in between 
those two this is refered to as mesoscopic scale. 
Once the scale is clear there is another freedom of choice in time and space descretization. 
For example the Cellular Automata approach is a model in discrete time and space. 
In the next chapter there shall be a selection of multiple models to give a greater insight into the workings of this topic in ressearch.

\newpage
\section{Approaches}
\subsection{Lattice Gas}
\subsection{Social Force}
\subsection{Fluid Dynamics}
\subsection{Agent based}
\subsection{Cellular Automata}

\newpage
\section{Cellular Automata}
\subsection{Random Walk}
The Random Walk is a discrete stochastic process. Here it is defined in a 2D lattice space consisting of cells. Every cell can be occupied by an agent. This is denoted by an encrease of the cells value of 1. 
Later in the works of the macroscopic modelling the values are often determined by a monte-carlo-method and get normalized. 
Sometimes in research this corresponding variable is mentioned as population-density.

A random walk in two dimensions is a mathematical model used to describe the movement of an object or particle that is randomly moving in two-dimensional space. The movement of the object is determined by a series of random steps in the horizontal (x) and vertical (y) directions.
In a simple random walk, each step is equally likely to be in any of the four cardinal directions (north, south, east, or west). The probability of the object moving in a particular direction is equal to 1/4. The distance the object moves in each step is often assumed to be constant, but it can also be a random variable.
The path of the object over a series of steps forms a random walk, which is a type of stochastic process. The behavior of the random walk can be studied using probability theory and statistical analysis.
One interesting aspect of random walks in two dimensions is that, despite the seemingly chaotic nature of the movement, there are patterns that emerge over time. For example, the object is more likely to be found further from the starting point as the number of steps increases. This is because the object has a greater chance of moving away from the starting point than it does of returning to it.
Another interesting property of random walks in two dimensions is that, on average, the object will return to its starting point after a large number of steps, regardless of the specific path it takes. This is known as the "drunkard's walk" phenomenon, as it is often used to model the movements of a drunken person trying to walk in a straight line.
Random walks in two dimensions have a wide range of applications, including modeling the movement of particles in gases and liquids, the spread of diseases, and even the behavior of financial markets.

As mentioned earlier every model has a underlying rulset. There are multiple versions for the Random Walk and its a non-trivial task to choose the right one for the specified purpose. 
Since we want to simulate the movement of humans at the top level of this work the ruleset has to be grounded in physical accuracy. 
At its core the Random Walks movement is defined by the neighboring cells a given agent can step into and the probability of which this cell is chosen. 
In the most used combination every agent can step onto a adjecent cell (not diagonally) with equal probabilities.
This alone leads to an undesired effect that all the cells can only be occupied exactly every second timestep an checkerboard pattern emerges and totally breaks any attempt of smoothing on a macroscopic scale. 
This is a well known effect in probability theory as well as in the field of stochastic processes.
So to prevent this the agent is also allowed to stay at its location. 
In this thesis we discuss multiple versions that build on top of that core ruleset. 

\subsection{Size Exclusion}
The main subject of this bachelor thesis is the comparison of the macroscopic behavior, when agents are allowed to overlap or not.
In the research of pedestrian dynamics this parameter is called size exclusion. In this chapter we want to define, and analyse this parameter and its concepts in depth. 
Size exclusion in the context of a random walk refers to the concept that the movement of an object or particle is restricted by the size of the space in which it is moving.
In a two-dimensional random walk, for example, the object may be confined to a grid of squares, with each square representing a unit of space. If the object is larger than a single square, it will not be able to move into squares that are already occupied by other objects or obstacles. This effectively limits the possible moves the object can make, and the random walk becomes constrained.
Size exclusion can also occur in three-dimensional space, such as when an object is moving through a network of interconnected tubes or channels. The object will be unable to move into spaces that are too small for it to fit through.
Size exclusion can have significant effects on the behavior of a random walk. For example, if the size of the object is much larger than the size of the squares in the grid, the random walk may be effectively confined to a small area and will not exhibit the expected long-term behavior, such as returning to the starting point after a large number of steps.
Size exclusion can also influence the rate at which an object moves through a space. If the object is able to move freely, it will likely have a higher average speed than if it is confined to a smaller area or restricted by obstacles.
In summary, size exclusion refers to the concept that the movement of an object in a random walk is limited by the size of the space in which it is moving. This can have significant effects on the behavior and speed of the random walk.
In this thesis we want to apply size exclusion by restricting agents to occupy the same space at any given time step.
Not only this but also the sequences of which the agents get updated makes a difference of the systems overall behaviour.
We want to discuss three different approaches.

\subsection{Static and dynamic field}


\newpage
\section{Random Walk Simulation}

\subsection{Sequentially updated agents}
The starting point would be to sequentially update a list of agents in a fixed order. This may be the most easy way to implement but comes with some undesired effects. 
Some agents further to the beginning of that list will always get prefered for reaching unoccupied cells. This may not lead to a huge difference in the macroscopic view of the system but struggles to accurately simulate the movement of one predetermined agent accurately.

\subsection{Scrambled-Sequentially updated agents}
To avoid the before mentioned inaccuracy we shuffle the order of which agents get updated. This may be simply implemented by scrambling the list of agents after every time step.
Since this is a well researched algorithm it will not add much complexity overhead to the simulation. 

\subsection{Parallel updated agents}
Finally we descuss a new approach of updating the system of agents. Like the name suggests we want to parallize the updating event such that every agent gets to choose a desired next location. After that there has to be a conflict solution implemented.
This is refered to as friction in the research and solves not only the problem of time complexity if well implemnented but also has a little advantage over other models since it can be interpreted as the cooperative behaviour of pedestrians. 
This conflict can be solved in multiple ways: 
\begin{itemize}
   \item No one gets to move. This is the most uncooperative scenario and can be adequate in panic situations.
   \item There is always someone that gets to move. This is the most cooperative version.
   \item Something in between. 
\end{itemize} 
In the macroscopic scale this friction parameter is introduced by a constant in the later discussed master equation.

\newpage
\section{Derivation of PDE for macroscopic scale}
The heat equatoion is a partial differantial equation that describes the movement of heat or energy in a given system.
It is a fundamental equation in the fields of engineering, physics and chemistry.

The heat equation has the form: 
\begin{equation}
\frac{\partial u}{\partial t} = \kappa \frac{{\partial}^2u}{{\partial x}^2}
\end{equation}
Where $u$ is the temperature of the system, $t$ is time, $x$ is position and $\kappa$ is the thermal conductivity of the material.
The heat equation can be used to solve for the temperature distribution in a given system at a particular time, or to predict the evolution of the temperature over time hence the name. It can also be used to determine the rate of heat transfer between two objects, such as during a collision or when one object is placed in contact with another.
Despite its natural purposes this equation also comes up in the research of pedestrian dynamics as the limit process of systems of random walk agents has a diffusive structure. Here we want to develop this macroscopic view on such systems.


\newpage
\section{Conclusion}
Until this point no conclusion can be made.

\cite*{kirchner2003friction}

\newpage





