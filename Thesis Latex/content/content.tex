%!TEX root = ../thesis.tex


\section{Introduction}
In this Bachelor thesis, we describe and analyse the behaviour of the Random Walk in discrete time and space components
and its limit process, which is key to some research of pedestrian dynamics.
This wide field tries to model behaviour of often multiple or thousands of entities, also refered to as agents, in a prediscribed timespace with an underlying 
ruleset (physics, updating, choices, size exclusion, etc.). After models are described they can be used to simulate, prognose, evaluate or prevent undesired reallife scenarios such as crowding, clugging, traffic jams etc. 
These models can sometimes be very computational demanding if the number of simulated agents or the resolution is surpasing some threshold. In some cases floating point precision is not neccesary to analyse the overall behaviour of a given system.
This is the reason why some models are based on a Cellular Automata approach.
This way the number of operations needed is redused while also keeping a high degree of realism in the results of the simulation. 
We want to discuss the applications of the random walk in such models as well as the differantiation between micro- and macroscopic modeling and how the preventation of overlapping agents can change the overall dynamics of such system. 

bullet points: 
pedestrian dynamics history
scales
models 
simulations
difficulties
current models  
topic of this bachelor thesis investigation in improvements of the model 

\section{Random Walk}
The Random Walk is a process often defined within 

\section{Size Exclusion}
The main subject of this bachelor thesis is the comparison of the macroscopic behavior, when agents are allowed to overlap or not.
In the research of pedestrian dynamics this parameter is called size exclusion. In this chapter we want to define, and analyse this parameter and its concepts in depth. 
\subsection*{Sequentially updated agents}
\cite[]{kafka2015prozess}



\section{Heat Equation}


\section{Simulations}


\section{Conclusion}





